\documentclass{article}
\textwidth 6.8in
\headheight -0.5in
%\flushbottom
\topmargin -0.5in
\oddsidemargin 0in
%\evensidemargin 1in
\textheight 10in
\pagestyle{empty}
\newcommand{\ol}{\overline}

\usepackage{amsmath,amsthm,amsfonts}

\usepackage{lmodern}
\usepackage[T1]{fontenc}
\usepackage[utf8]{inputenc}
\usepackage{textcomp}
\usepackage[slovak]{babel}

\begin{document}
Computer algebra systems \\
Cela tato sekcia hovori o tom, preco je vyhodne pouzivat pocitac na riesenie velkych systemov rovnic.
Jednym z hlavnych dovodov je to, ze clovek je omylny a moze spravit chybu aj v jednotlivom vypocte a pri takomto velkom pocte rovnic je mozne,
z k tejto chyba aj dojde. Dalsou velkou vyhodou je to, ze zdanlivo dlhy vypocet pre cloveka dokaze pocitac vyriesit za zanedbatelne dlhy cas.
Myslim, ze pocitanie systemov rovnic za pomoci pocitaca urcite v buducnosti vyuzijem a to hlavne vtedy ak sa budem venovat pocitacovej grafike,
kde sa prave taketo systemy pouzivaju. Dalsiou moznostou je vyuzitie v oblasti umelej inteligencie teda presnejsie pri vypoctoch v ramci neuronovych sieti.
V neposlednom rade to pravdepodobne pouzijem aj v dalsom fungovani na fakulte, ved preco by sme pocitali nieco pracne rucne ked nam to vie neomylne a rychlo spocitat pocitac :) \\
Nie som si isty, ci som nasledujuce ulohy pochopil spravne ale pise sa v nich, ze mame na vypocet pouzit pocitac. Nepise sa vsak ci ma byt SW nas alebo mozme pouzit uz nejaky funkcny.
\begin{itemize}
    \item 1 \begin{itemize}
        \item a\\
                vysledok pomocou programu wolfram: h = 1, c = 4
        \item b\\
                vysledok: h = k/3; i = k; j = k/3
    \end{itemize}
    \item 4 \begin{itemize}
        pre tuto ulohu mi pocitac (Wolfram) povedal, ze rovnica cislo 2 je (invalid)
    \end{itemize}
\end{itemize}
\end{document}
