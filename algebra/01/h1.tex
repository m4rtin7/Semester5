\documentclass{article}
\textwidth 6.8in
\headheight -0.5in
%\flushbottom
\topmargin -0.5in
\oddsidemargin 0in
%\evensidemargin 1in
\textheight 10in
\pagestyle{empty}
\newcommand{\ol}{\overline}

\usepackage{amsmath,amsthm,amsfonts}

\usepackage{lmodern}
\usepackage[T1]{fontenc}
\usepackage[utf8]{inputenc}
\usepackage{textcomp}
\usepackage[slovak]{babel}

\begin{document}
%\begin{large}
%\begin{bf}
\begin{center}

{\bf Line\' arna algebra} 

{\sc Dom\' aca \' uloha  \#1.}

{\sc Martin Zavadzan}


\end{center}

\begin{enumerate}
\item 


\begin{description}
\item{a)} 
\[ \begin{pmatrix}
    1 & -2 & -5 &  1 & -13 & |& 12 \\
    1 & -2 & -5 &  1 &  0 & | & 1 \\
    2 & -4 & -10 &  2 & -1 & | & 11  
\end{pmatrix} 
\sim
\begin{pmatrix}
    1 & -2 & -5 &  1 & -13 & |& 12 \\
    0 & 0 & 0 &  0 &  13 & | & -11 \\
    0 & 0 & 0 &  0 & 25 & | & -13  
\end{pmatrix} 
\sim
\begin{pmatrix}
    1 & -2 & -5 &  1 & -13 & |& 12 \\
    0 & 0 & 0 &  0 &  1 & | & - \frac{11}{13}\\[6pt]
    0 & 0 & 0 &  0 & 0 & | &   \frac{106}{13}
\end{pmatrix}
\]
Tento system nema riesenie ako mozem vidiet podla posledneho riadka, kedze 0 $\neq \frac{106}{13}$


\item{b)}
\[ \begin{pmatrix}
1 & -  2 & 3 &  1 & -2 & | & 12 \\
2& -  2 & 2 &  1 & 0 & | & 1 \\
3 & -  4 & 5 &  2 & -2 & | & 13 
\end{pmatrix}
\sim
\begin{pmatrix}
1 & -2 & 3 &  1 & -2 & | & 12 \\
0 & 2 & -4 &  -1 & 4 & | & -23 \\
0 & 2 & -4 &  1 & 4 & | & -23 
\end{pmatrix}
\sim \]
\[
\begin{pmatrix}
1 & -2 & 3 &  1 & -2 & | & 12 \\
0 & 1 & -2 &  -0,5 & 2 & | & -11,5 \\
0 & 0 & 0 &  0 & 0 & | & 0 
\end{pmatrix}
\]
Tento system ma nekonecne vela rieseni a to su tie, ktore vyhovuju nasledovnym podmienkam:\\
$x_3 = p, x_4 = q, x_5 = r$ kde $p, q, r \in R$, potom:
\begin{align}
    x_1 - 2x_2 + 3p + q - 2r = 12 \\
    x_1 = 2x_2 - 3p - q + 2r + 12 \\ \\
    x_2 - 2p - 0,5q + 2r = -11,5\\
    x_2 = 2p + 0,5q - 2r -11,5
\end{align}}
a teda, riesenie je
\begin{align}
    \begin{pmatrix}
        x_1\\
        x_2\\
        x_3\\
        x_4\\
        x_5
    \end{pmatrix}
    =
    \begin{pmatrix}
        2x_2 - 3p - q + 2r + 12\\
        2p + 0,5q - 2r -11,5\\
        p\\
        q\\
        r
    \end{pmatrix}
\end{align}
\item{c)}
\[ \begin{pmatrix}
1& -2 &  -5 & 1 & -2 & | & 12 \\
1 & -2 &  -5 & 1 & 0 & | & 1 \\
0 & -4 & -10 & 2 & -2 & | &  11 \\
11 & 0 & -5 & 1 & -2 & | & 11 \\
1 & -2 & -5 & 0 & -2 & | & 12
\end{pmatrix}
\sim
\begin{pmatrix}
    1& -2 &  -5 & 1 & -2 & | & 12 \\
    0 & 0 &  0 & 0 & 2 & | & -11 \\
    0 & -4 & -10 & 2 & -2 & | &  11 \\
    0 & 22 & 50 & 10 & 20 & | & -121 \\
    0 & 0 & 0 & -1 & 0 & | & 0
\end{pmatrix}
\sim
\]

\[
\begin{pmatrix}
    1& -2 &  -5 & 1 & -2 & | & 12 \\
    0 & 1 & 2,5 & -0,5 & 0,5 & | &  -2,75 \\
    0 & 22 & 50 & -10 & 20 & | & -121 \\
    0 & 0 & 0 & 1 & 0 & | & 0 \\
    0 & 0 &  0 & 0 & 1 & | & -5,5
\end{pmatrix}
\sim
\begin{pmatrix}
    1& -2 &  -5 & 1 & -2 & | & 12 \\
    0 & 1 & 2,5 & -0,5 & 0,5 & | &  -2,75 \\
    0 & 0 & -5 & 1 & 9 & | & -60,5 \\
    0 & 0 & 0 & 1 & 0 & | & 0 \\
    0 & 0 &  0 & 0 & 1 & | & -5,5
\end{pmatrix}
\sim
\]
\[
\begin{pmatrix}
    1& -2 &  -5 & 1 & -2 & | & 12 \\
    0 & 1 & 2,5 & -0,5 & 0,5 & | &  -2,75 \\
    0 & 0 & 1 & -0,2 & -1,8 & | & 12,1 \\
    0 & 0 & 0 & 1 & 0 & | & 0 \\
    0 & 0 &  0 & 0 & 1 & | & -5,5
\end{pmatrix}
\sim
\begin{pmatrix}
    1& -2 &  -5 & 1 & 0 & | & 1 \\
    0 & 1 & 2,5 & -0,5 & 0 & | &  0 \\
    0 & 0 & 1 & -0,2 & 0 & | & 2,2 \\
    0 & 0 & 0 & 1 & 0 & | & 0 \\
    0 & 0 &  0 & 0 & 1 & | & -5,5
\end{pmatrix}
\sim
\]
\[
\begin{pmatrix}
    1& -2 &  -5 & 0 & 0 & | & 1 \\
    0 & 1 & 2,5 & 0 & 0 & | &  0 \\
    0 & 0 & 1 & 0 & 0 & | & 2,2 \\
    0 & 0 & 0 & 1 & 0 & | & 0 \\
    0 & 0 &  0 & 0 & 1 & | & -5,5
\end{pmatrix}
\sim
\begin{pmatrix}
    1& -2 &  0 & 0 & 0 & | & 12 \\
    0 & 1 & 0 & 0 & 0 & | &  -5,5 \\
    0 & 0 & 1 & 0 & 0 & | & 2,2 \\
    0 & 0 & 0 & 1 & 0 & | & 0 \\
    0 & 0 &  0 & 0 & 1 & | & -5,5
\end{pmatrix}
\sim
\begin{pmatrix}
    1& 0 &  0 & 0 & 0 & | & 1 \\
    0 & 1 & 0 & 0 & 0 & | &  -5,5 \\
    0 & 0 & 1 & 0 & 0 & | & 2,2 \\
    0 & 0 & 0 & 1 & 0 & | & 0 \\
    0 & 0 &  0 & 0 & 1 & | & -5,5
\end{pmatrix}
\]
Uloha ma prave jedno riesenie, kde 
\[
\begin{pmatrix}
    x_1 \\
    x_2 \\
    x_3 \\
    x_4 \\
    x_5
\end{pmatrix}
=
\begin{pmatrix}
    1 \\
    -5,5 \\
    2,2 \\
    0 \\
    -5,5
\end{pmatrix}
\]

\end{description}
\end{enumerate}
%\end{bf}
%\end{large}
\end{document}


\end{description}
\end{document}

\end{description}
%\end{bf}
%\end{large}
\end{document}
